\begin{figure}[ht]
    \centering
    \begin{subfigure}[b]{0.4\textwidth}
    \centering 
    \includegraphics[width=\linewidth]{img/chap3/robot.pdf}
    \end{subfigure}
    \begin{subfigure}[b]{0.59\textwidth}
    \centering 
    \includegraphics[width=\linewidth]{img/chap3/legs.pdf}
    \end{subfigure}
    \caption{Graphical overview of soft quadruped robot model, SoftQ and the Compressible Tendon-driven Soft Actuator (CTSA). (a) Rendered robot with key state notations. (b) The structure and notations of the CTSA. The upper part from section A-A is the rigid thigh for maintaining the actuator length and the lower part is compressible and bendable. The bending angle of the lower part is noted as αb. (c) The top view of the bent CTSA of section A-A. αr refers to the rotational angle of the CTSA. (d) The CTSA compression is realized by pulling the three tendons by the same amount, with the compression length being noted as $z_l$., originated from Li et al.\cite{jiSynthesizingOptimalGait2022, jiOmnidirectionalWalkingQuadruped2022}}
    \label{fig:robot}
\end{figure}