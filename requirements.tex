\section*{Requirements}
\subsection*{Abstract and Introduction} 
Concise overview of the research field relevant to the Master’s thesis, and identification of important open questions. Clear description of the problem addressed in the thesis, and clear statement of the project goals. Appropriate literature citations in a uniform and consistent style and format (applies to whole thesis)

\subsection*{Methods and Results}

Clear and complete description of the methods used such that all experiments can be reproduced by others. Clear description of the logic and hypotheses underlying the choice of performed experiments. Clear and complete presentation and correct interpretation of the experimental results. Appropriate quality and quantity of results. Correct presentation and labelling of figures and tables.

\subsection*{Discussion and Conclusions} 

Concise discussion and critical evaluation of the obtained results with respect to the original goals. Discussion of the results into a more general context within the research field. Discussion of the possibilities and limitations of the applied experimental techniques. Formulation of new hypotheses, outlook for future work.

\subsubsection{Abstract}

An abstract that allows an "informed reader" to grasp the contents of the report. This abstract should be 100\% finished at the 80\%-draft of the report.

\subsubsection{Introduction}

Define the background of your project, the purpose of the report, specific descriptions of problems, your research questions, any limitations to the scope of your investigation, and (briefly) your chosen methodology.

\subsubsection{Related Work / Theoretical Background / Theoretical Framework}

It is necessary to describe the relevant, scientific background knowledge concerning the area in which you will perform your thesis work. One goal of this section is to analyse the literature reviewed and thus specify the direction of your project work. The main goal is for you to build on and expand your existing knowledge to assist you in dealing with the task at hand. This section is ideally structured as describing related work (scientific papers that perform variants of the same investigation as you do), as a theoretical background (scientific papers that deal with the same problem area as you do), or as a theoretical framework (scientific papers that you use to create a theoretical model for explaining your results).

\subsubsection{Methodology}

Describe the scientific methodology that you have used in your investigation, using an appropriate scientific textbook as a reference. You have been introduced to case studies and experiments in the research methodology course, but you can use other methodologies if you can motivate it. For a case study this section includes a detailed description if your chosen case(s). An experimental study can provide hypotheses here, or in Section B (whichever is most easily read). This section also includes a description of the choices you have made to support the internal validity, external validity and reliability of your investigation.
Note that methodologies can be described as being quantitative or qualitative, but these two words do not describe any specific methodology by themselves.

\subsubsection{Result/Analysis}

Present the results from your study. If appropriate for readability, this section can succinctly compare your empirical results with existing theory in the field and/or your hypotheses. This interpretation/analysis should then be directly related to the papers described in Section B.

\subsubsection{Discussion}

In this section you discuss your findings. This discussion should answer your research questions in relation to the papers described in Section B, and possibly other relevant papers. In other words, this section should put your results into perspective considering the results from other studies, the wider discussion in your problem area and/or considering the theoretical models you have used.
Note that this can include an evaluation of your results to highlight e.g., performance improvements of a solution you have investigated, but that this is not enough on its own – all improvements must be put into perspective considering relevant aspects of the discussion in associated papers.
This section can also include a “limitations” subsection, which describes limitations to your findings - such as when it is not appropriate to trust them.
This section should also include a discussion of your results and investigation in relation to ethical, social or sustainability issues.

\subsubsection{Conclusions and Future Work}

Describe the conclusions of your work and give recommendations on how to proceed with the work.

\subsubsection{Appendices}

Important, but complementary material/results can be placed in appendices. This includes details of any implementation (practical work stages, etc.), large data sets, etc.”