\newpage
\thispagestyle{empty}
~\\
\vfill
{ \setstretch{1.1}
	\subsection*{Authors}
	NIU Xuezhi { }$\langle$\href{mailto:xuezhin@kth.se}{xuezhin@kth.se}$\rangle$\\
    Msc. in Engineering Design - Mechtronics \\
	KTH Royal Institute of Technology
	
	\subsection*{Place for Project}
	Stockholm, Sweden
	
	\subsection*{Examiner}
	Lei Feng \\
	Department of Machine Design \\
	KTH Royal Institute of Technology
	
	\subsection*{Supervisor }
	Tan Kaige\\
 % Mechatronics and Embedded Control Systems division, 
    Department of Machine Design\\
	KTH Royal Institute of Technology
	~
}

\newpage
\titleformat
{\chapter} % command
[display] % shape
{\normalfont\huge\bfseries} % format
{\hfill Chapter \ \thechapter} % label
{-2ex} % sep
{
    % \noindent
    \makebox[0pt][l]{\rule[.6ex]{\linewidth}{2.5pt}}%
    \rule[.3ex]{\linewidth}{.6pt}
    % \vspace{-0.5ex}
} % before-code
[
\vspace{-2ex}%
\rule{\textwidth}{.6pt}
] % after-code

% \titleformat
% {\chapter} % command
% [display] % shape
% {\normalfont\huge\bfseries} % format
% {\hfill \thechapter} % label
% {-2ex} % sep
% {
%     \vspace{-1.2em}
%     \begin{table}[hp]
    \centering
    {\sffamily
    \resizebox{\textwidth}{!}{%
    \begin{tabular}{|cp{5cm}p{5cm}|}
        \hline
        \multicolumn{1}{|p{3.875cm}}{\begin{tabular}[l]{@{}c@{}}\includegraphics[width=0.5\linewidth]{setup/img/kth\_logo.eps}\\ \tiny\sffamily\bfseries KTH Industriell Teknik \vspace*{-0.3cm}\\ \tiny\sffamily\bfseries och Management\end{tabular}} &
          \multicolumn{2}{c|}{\begin{tabular}[c]{@{}c@{}}\textbf{Examensarbete TRITA-ITM-EX 2023:XXX}\\ \\ \\ \textbf{Adaptiv gångstyrning av en mjuk fyrbent robot} \\ \textbf{genom modellbaserad förstärkningsinlärning}\vspace{2em}\end{tabular}} \\
        \multicolumn{1}{|c}{} &
          \multicolumn{2}{c|}{NIU Xuezhi} \\ \hline
        \multicolumn{1}{|l|}{\begin{tabular}[c]{@{}l@{}}\scriptsize Godkänt\\ \normalsize 2023-mån-dag\end{tabular}} &
          \multicolumn{1}{p{5cm}|}{\begin{tabular}[c]{@{}l@{}}\scriptsize Examinator\\ \normalsize Lei Feng\end{tabular}} &
          \begin{tabular}[c]{@{}l@{}}\scriptsize Handledare\\ \normalsize Tan Kaige\end{tabular} \\ \hline
        \multicolumn{1}{|p{5cm}|}{} &
          \multicolumn{1}{l|}{\begin{tabular}[c]{@{}l@{}}\scriptsize Uppdragsgivare\\ \normalsize Lei Feng\end{tabular}} &
          \begin{tabular}[c]{@{}l@{}}\scriptsize Kontaktperson\\ \normalsize Lei Feng\end{tabular} \\ \hline
    \end{tabular}%
    }}
\end{table}

%     \vspace{-1em}
%     % \noindent
%     \makebox[0pt][l]{\rule[.6ex]{\linewidth}{2.5pt}}%
%     \rule[.3ex]{\linewidth}{.6pt}
%     % \vspace{-0.5ex}
% } % before-code
% [
% \vspace{-2ex}%
% \rule{\textwidth}{.6pt}
% ] % after-code

% \newpage
% %%%%%%%%%%%%%%%%%%%%%%%%%%%%%%%%%%%%
% %%	 The Swedish abstract         %%
% %%%%%%%%%%%%%%%%%%%%%%%%%%%%%%%%%%%%
% \chapter*{Sammanfattning}
% %%%%%%%%%%%%%%%%%%%%%%%%%%%%%%%%%%%%

% Fyrbenta robotar har fördelen att kunna manövrera i komplex terräng utan mänsklig ansträngning, och de kan till exempel ge större absorptionsförmåga så att roboten bättre kan stå emot stötar och slag. Även om stela robotar är kända för sin snabba respons och exakta rörelser, kan de vara begränsade i sitt rörelseområde på grund av sina stela mekaniska strukturer, såsom leder och hydrauliska ställdon. På senare tid har utvecklingen av sensorer, ställdon och datorer gjort det lättare att styra mjuka fyrbenta robotar i verkliga livet, vilket gör det möjligt att utforska styrstrategier för ökad anpassningsförmåga och effektivitet i olika terränger och miljöer. Samtidigt underlättar användningen av förstärkningsinlärning för styrning av robotteknik uppnåendet av autonoma och anpassningsbara robotbeteenden, där robotar lär sig att agera genom interaktioner med sina miljöer för att optimera prestanda. 

% Trots framstegen inom styrning av fyrbenta robotar kräver integrering av avancerade styrmetoder som förstärkningsinlärning omfattande utbildning och robusthetsöverväganden för att säkerställa säker funktionalitet i den verkliga världen. Att anpassa effektiva simuleringsbaserade styrtekniker till konkreta robotar stöter på utmaningar som härrör från skillnaden mellan simulerade och verkliga miljöer, vilket kräver förbättrade metoder för förstärkningsinlärning. Denna avhandling strävar efter att bidra väsentligt till att öka robotförmågan genom att använda en optimal gångkontroll av mjuka fyrhjuliga robotar genom modellbaserad förstärkningsinlärning. Genom att ta itu med de utmaningar som är kopplade till den verkliga implementeringskomplexiteten och förfina kontrollstrukturerna, syftar den till att bana en transformativ väg för en harmonisk sammanslagning av förstärkningsinlärning i den mjuka fyrhjuliga robotdomänen, vilket förbättrar prestanda, anpassningsförmåga och autonomi.

% Inom innehållet i denna rapport ingår implementeringen av modellbaserade tekniker för förstärkt inlärning för att optimera gångkontrollen, med detaljer om simuleringsinställningen, belöningsstrukturen och policyförfiningsprocessen. En omfattande presentation av resultaten tillhandahålls, tillsammans med en djupgående analys som validerar effektiviteten hos de föreslagna metoderna. Analysen visar på betydande förbättringar av träningseffektiviteten och det autonoma beteendet, vilket i slutändan bekräftar effektiviteten hos de föreslagna metoderna. 

% \subsection*{Nyckelord}
% Fyrhjuliga robotar, mjuk robotteknik, adaptiv gångkontroll, modellbaserad förstärkningsinlärning


% \pagestyle{plain}
% \titleformat
% {\chapter} % command
% [display] % shape
% {\normalfont\huge\bfseries} % format
% {\hfill \thechapter} % label
% {-2ex} % sep
% {
%     \vspace{-1.2em}
%     \selectfont
\begin{table}[hp]
    \centering
    {\sffamily
    \resizebox{\textwidth}{!}{%
    \begin{tabular}{|cp{5cm}p{5cm}|}
        \hline
        \multicolumn{1}{|p{3.875cm}}{\begin{tabular}[l]{@{}c@{}}\includegraphics[width=0.5\linewidth]{setup/img/kth\_logo.eps}\\ \tiny\bfseries KTH Industrial Engineering \vspace*{-0.3cm}\\ \tiny\bfseries and Management\end{tabular}} &
          \multicolumn{2}{c|}{\begin{tabular}[c]{@{}c@{}}\textbf{Master of Science Thesis TRITA-ITM-EX 2023:XXX}\\ \\ \\ \textbf{Adaptive Gait Control of Soft Quadruped Robot} \\ \textbf{by Model-based Reinforcement Learning}\vspace{2em}\end{tabular}} \\
        \multicolumn{1}{|c}{} &
          \multicolumn{2}{c|}{NIU Xuezhi} \\ \hline
        \multicolumn{1}{|l|}{\begin{tabular}[c]{@{}l@{}}\scriptsize Approved\\ \normalsize 2023-month-day\end{tabular}} &
          \multicolumn{1}{p{5cm}|}{\begin{tabular}[c]{@{}l@{}}\scriptsize Examiner\\ \normalsize Lei Feng\end{tabular}} &
          \begin{tabular}[c]{@{}l@{}}\scriptsize Supervisor\\ \normalsize Tan Kaige\end{tabular} \\ \hline
        \multicolumn{1}{|p{5cm}|}{} &
          \multicolumn{1}{l|}{\begin{tabular}[c]{@{}l@{}}\scriptsize Commissioner\\ \normalsize Lei Feng\end{tabular}} &
          \begin{tabular}[c]{@{}l@{}}\scriptsize Contact person\\ \normalsize Lei Feng\end{tabular} \\ \hline
    \end{tabular}%
    }}
\end{table}

%     \vspace{-1em}
%     % \noindent
%     \makebox[0pt][l]{\rule[.6ex]{\linewidth}{2.5pt}}%
%     \rule[.3ex]{\linewidth}{.6pt}
%     % \vspace{-0.5ex}
% } % before-code
% [
% \vspace{-2ex}%
% \rule{\textwidth}{.6pt}
% ] % after-code

\newpage
\pagenumbering{Roman}
% \thispagestyle{plain}
%%%%%%%%%%%%%%%%%%%%%%%%%%%%%%%%%%%%
%%  The English abstract          %%
%%%%%%%%%%%%%%%%%%%%%%%%%%%%%%%%%%%%
\chapter*{Abstract}
%%%%%%%%%%%%%%%%%%%%%%%%%%%%%%%%%%%%

Quadruped robots offer advantages in navigating complex terrain without human intervention. They possess greater shock absorption capabilities, allowing them to withstand impacts effectively. In contrast, rigid robots, known for their rapid response and precise motion, often face limitations in their range of motion due to their inflexible mechanical components, such as joints and hydraulic actuators. Recent advancements in sensors, actuators, and computing have simplified real-world control of soft quadruped robots, enabling the development of control strategies to enhance their adaptability and efficiency across various terrains and environments. Additionally, the use of reinforcement learning in robotics control facilitates the achievement of autonomous and adaptable behaviors, as robots learn to optimize their performance through interactions with their surroundings.

However, despite advancements in quadruped robot control, the integration of advanced control methods like reinforcement learning requires extensive training and robustness considerations to ensure safe real-world operation. Adapting effective simulation-based control techniques to physical robots faces challenges arising from differences between simulated and real-world environments, necessitating improvements in reinforcement learning methodologies. This thesis aims to make a substantial contribution to enhancing robotic capabilities by employing optimal gait control for soft quadruped robots through model-based reinforcement learning. By addressing the complexities of real-world implementation and refining control structures, it seeks to facilitate the seamless integration of reinforcement learning into the field of soft quadruped robotics, thereby improving performance, adaptability, and autonomy.

This report includes the implementation of model-based reinforcement learning techniques for optimizing gait control, providing details on the simulation setup, reward system, and policy refinement process. It also offers a comprehensive presentation of the results, accompanied by an in-depth analysis that confirms the effectiveness of the proposed methods. This analysis demonstrates significant enhancements in training efficiency and autonomous behavior, validating the efficacy of the approaches taken.

\vspace{2ex}
\subsection*{Keywords:}
Quadruped Robots, Soft Robotics, Reinforcement Learning, Gait Control, Model-Based Control Optimization




\newpage
\chapter*{Acknowledgements}
I am deeply grateful for the opportunity to complete this Master's Thesis and would like to acknowledge the support and guidance of several individuals who have contributed to the successful completion of this project.

First and foremost, I would like to extend my sincere thanks to the Mechatronics Department of Machine Design at KTH for their unwavering support throughout the duration of this project. I am particularly indebted to my academic supervisor, Mr. Tan Kaige, for his guidance and unwavering commitment to keeping me on track. Moreover, I would like to express my deep appreciation to the examiner, Prof. Lei Feng, for his interest in my work and for his valuable consultation. Dr. Fredrik Asplund deserves my gratitude for his invaluable guidance and advice regarding the research methodology, which helped shape the initial direction of this thesis work. In addition, I appreciate the assistance of Seshagopalan Thorapalli Muralidharan in identifying and resolving potential issues related to the electrical board and wiring. 

I cannot overlook the unwavering support of my parents throughout my academic journey, and I deeply appreciate their encouragement, which has been a constant source of motivation. Furthermore, I would like to recognize the contribution of ChatGPT, the language model developed by OpenAI, for its invaluable assistance in reducing the written content in this thesis. You saved my life.

Lastly, I would also like to acknowledge myself, as I am also proud of the dedication, perseverance, and hard work I have put into completing this Master's Thesis.

\vspace{2cm}
\hfill NIU Xuezhi 牛雪芝

\hfill Stockholm, \monthname{ }2023 $\ddot\smile$

\newpage

\chapter*{Acronyms}

\section*{Abbreviations}
\begin{acronym}[RDBMS]
\acro{3D}{Three Dimensional}
\acro{4D}{Four Dimensional}
\acro{DoF}{Degree of Freedom}
\acro{RL}{Reinforement Learning}
\acro{MBRL}{Model-Based Reinforement Learning}
\acro{MFRL}{Model-Free Reinforement Learning}
\end{acronym}

\section*{General mathematics}




\newpage

\etocdepthtag.toc{mtchapter}
\etocsettagdepth{mtchapter}{subsection}
\etocsettagdepth{mtappendix}{none}

\tableofcontents
\newpage