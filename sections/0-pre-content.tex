\newpage
\thispagestyle{plain}
~\\
\vfill
{ \setstretch{1.1}
	\subsection*{Authors}
	NIU Xuezhi { }\href{mailto:xuezhin@kth.se}{\faEnvelope xuezhin@kth.se}\\
    Msc. in Engineering Design - Mechtronics \\
	KTH Royal Institute of Technology
	
	\subsection*{Place for Project}
	Stockholm, Sweden
	
	\subsection*{Examiner}
	Lei Feng \\
	Department of Machine Design \\
	KTH Royal Institute of Technology
	
	\subsection*{Supervisor }
	Tan Kaige\\
 % Mechatronics and Embedded Control Systems division, 
    Department of Machine Design\\
	KTH Royal Institute of Technology
	~
}


\newpage
\thispagestyle{plain}
%%%%%%%%%%%%%%%%%%%%%%%%%%%%%%%%%%%%
%%  The English abstract          %%
%%%%%%%%%%%%%%%%%%%%%%%%%%%%%%%%%%%%
\chapter*{Abstract}
%%%%%%%%%%%%%%%%%%%%%%%%%%%%%%%%%%%%

Quadruped robots shared advantages of maneuverability in complex terrain without human effort, for instance, they can provide greater absorption capacity, allowing the robot to better withstand impacts and shocks. While rigid robots are known for their fast response and motion accuracy, they may be limited in their motion range due to their rigid mechanical structures, such as joints and electric motors. Recently, advances in sensors, actuators, and computers have made it possible to control soft quadruped robots in real life easily.

\vspace{2cm}
Write an abstract. Introduce the subject area for the project and describe the problems that are solved and described in the thesis. Present how the problems have been solved, methods used and present results for the project. Use probably one sentence for each chapter in the final report.

The presentation of the results should be the main part of the abstract. Use about ½ A4-page.
English abstract




\subsection*{Keywords}
Machine Learning, \ac{MBRL}, Soft Quadruped Robots, Gait Pattern





\newpage
\thispagestyle{plain}
%%%%%%%%%%%%%%%%%%%%%%%%%%%%%%%%%%%%
%%	 The Swedish abstract         %%
%%%%%%%%%%%%%%%%%%%%%%%%%%%%%%%%%%%%
\chapter*{Abstract}
%%%%%%%%%%%%%%%%%%%%%%%%%%%%%%%%%%%%
Mjuka fyrbenta robotar har potential att övervinna begränsningarna hos styva robotar i komplexa terrängmiljöer. De kan utnyttja sin större deformationskapacitet för att anpassa sig till stötar och chocker utan mänsklig intervention. Styva robotar, å andra sidan, kan ha hög hastighet och precision, men deras rörelseomfång kan vara begränsat av deras styva mekaniska komponenter, såsom leder och elmotorer. Nya framsteg inom sensorik, ställdonsteknik och beräkning har möjliggjort effektiv styrning av mjuka fyrbenta robotar i realtid.


Skriv samma abstract på svenska. Introducera ämnet för projektet och beskriv problemen som löses i materialet. Presentera 

\subsection*{Nyckelord}
Maskininlärning, Modellbaserad Förstärkningsinlärning, Mjuka Fyrbenta Robotar, Gångmönster


\newpage
\thispagestyle{plain}
\chapter*{Acknowledgements}
Write a short acknowledgements. Don't forget to give some credit to the examiner and supervisor.

\vspace{2cm}
\hfill NIU Xuezhi 

\hfill Stockholm, \monthname{ }2023 $\ddot\smile$

\newpage

\chapter*{Acronyms}

\section*{Abbreviations}
\begin{acronym}[RDBMS]
\acro{3D}{Three Dimensional}
\acro{4D}{Four Dimensional}
\acro{DoF}{Degree of Freedom}
\acro{RL}{Reinforement Learning}
\acro{MBRL}{Model-Based Reinforement Learning}
\acro{MFRL}{Model-Free Reinforement Learning}
\end{acronym}

\section*{General mathematics}




\newpage

\etocdepthtag.toc{mtchapter}
\etocsettagdepth{mtchapter}{subsection}
\etocsettagdepth{mtappendix}{none}
\thispagestyle{plain}
\tableofcontents

\newpage


