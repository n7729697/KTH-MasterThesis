
\chapter{Introduction}
\label{chap1}
\textit{In this chapter, the background of soft quadruped robots and reinforcement learning approaches to its control is presented. Formulated research questions are listed together with methodologies. In addition, the limitations and delimitation to this thesis are discussed, as well as the ethics and sustainability analysis.}

\section{The Background}
In the realm of engineering, robotics emerges as a magnificent confluence of mechanical, electrical, and computer science, orchestrating a symphony of autonomous systems designed to push the boundaries of human potential and expanding efficiency\cite{billardTrendsChallengesRobot2019}. The involvement of robotic systems in our everyday lives has become increasingly commonplace, with its presence being felt across diverse fields such as manufacturing\cite{wangCurrentResearchesFuture2018}, agriculture\cite{liDevelopmentFieldEvaluation2023}, transportation\cite{zhangFindingCriticalScenarios2023}, education\cite{riedoThymioIIRobot2013} and even personal assistance\cite{openaiGPT4TechnicalReport2023}. In the domain of mobile robotics, the conventional approach to locomotion has been through the usage of wheels or tracks, making them apt for navigation on smooth surfaces\cite{liResearchMammalBionic2011}. Nonetheless, when it comes to maneuvering through unstructured and hazardous environments, such as the ones often encountered during search and rescue missions\cite{hawkesSoftRobotThat2017}, industrial production lines\cite{huDesignQuadrupedInspection2021}, or scientific research endeavors\cite{hewingLearningbasedModelPredictive2020}, legged robots, characterized by their flexible structures, have demonstrated their worth. However, recent advancements in material sciences and design have given rise to a new breed of robots known as soft robots. These robots, owing to their deformable structure, have the unique ability to mold themselves according to their surroundings, which makes them ideal for interacting with humans or fragile objects in a safe manner\cite{muralidharanSoftQuadrupedRobot2021}. This necessitates the development of advanced control strategies that can adapt to the dynamic and ever-changing morphology of these robots\cite{wangControlStrategiesSoft2022}. This thesis project seeks to investigate innovative control strategies that enable effective motion control of soft quadruped robots, contributing to the advancement of soft robotics technology.

In the previous studies\cite{thorapallimuralidharanContinuumActuatorBased2020} completed by the KTH Mechatronics and Embedded Control Systems Unit, tendon-driven soft continuum actuators were implemented in \ac{3D} /\ac{4D} printed structures to operate as legs of quadruped robots. Specifically, the tendons are used to transmit the force from motors, while the soft material acts as the core and the rigid disc acts as the tendon guide to form the actuator body. Since soft legs are inherently compliant to the terrain, they have had an increased capability of traversing complicated environments. Based on this soft leg, a soft quadrupedal robot prototype was developed and gait analysis was performed to enable the robot to walk\cite{daneliaStructureGaitOptimizationof2021}, namely SoftQ. Modelling for the actuators and legs are quite important for closed loop control, since the robots will benefit from feedback to the closed loop control when interacting with the terrine. Therefore, the KTH team\cite{muralidharanSoftQuadrupedRobot2021} have already investigated the modeling process of a quadruped robot enabled by four tendon-driven continuum actuators on MathWorks Simulink\textsuperscript{\textregistered}. Based on the developed soft robot simulation model, a gait controller was developed by a \ac{RL} algorithm \ac{SAC} recently\cite{jiSynthesizingOptimalGait2022}.

\ac{RL} has shown great potential in studying the control of quadruped robot motion due to its ability to learn complex control policies through trial-and-error interactions with the environment\cite{rechtTourReinforcementLearning2019}. This approach is particularly relevant for soft quadruped robots, as their continuous and deformable morphology poses significant challenges for traditional control methods\cite{zhangEffectiveSoftRobot2017}. Moreover, \ac{RL} enables the robot to learn from experience, allowing it to optimize its behavior based on feedback received from the environment in the form of a reward signal, thereby leading to the development of more efficient and robust control policies that can handle complex and unpredictable environments\cite{jiLearningbasedControl4D2022}. Consequently, \ac{RL} is ideally suited for studying the control of complex systems such as soft quadruped robots, which are challenging to model and control using traditional methods\cite{rechtTourReinforcementLearning2019}. Although various \ac{RL} algorithms have been used to develop policies for quadruped robots\cite{cebeOnlineDynamicTrajectory2021,chignoliOnlineTrajectoryOptimization2021,chignoliRapidReliableQuadruped2022}, challenges remain in developing RL-based gait control strategies. These strategies need to handle the continuous and deformable morphology of the robot while also being computationally efficient\cite{wangEfficientLearningRobust2022}.  Hence, gait control of soft quadruped robots stands as a promising area of research with the potential to advance the field of soft robotics. However, the traditional learning methods suffer from challenges such as high sample complexity\cite{haarnojaSoftActorCriticOffPolicy2018}, instability during training\cite{zhangUnderstandingDeepLearning2021}, and the need for careful tuning of hyperparameters\cite{haarnojaSoftActorCriticAlgorithms2019}. There is a clear need for further research to develop more efficient and effective RL control strategies that can handle the complexity of these robots and enable them to perform tasks in real-world environments\cite{annaswamyAdaptiveControlIntersections2023}. Looking ahead, \ac{RL} is likely to continue being applied for learning dynamic walking gaits for quadruped robots in both simulated and real-world environments. Moreover, \ac{RL} is expected to be utilized for controlling the behavior of quadruped robots in various tasks, including but not limited to obstacle avoidance and terrain adaptation. These applications of \ac{RL} hold great promise for advancing the field of robotics gait control and improving the efficiency and effectiveness of soft quadruped robots in various practical applications.

In short, the core of this thesis lies at the intersection of two rapidly developing research areas, soft-bodied robotics and reinforcement learning. Incorporating \ac{MBRL}, which involves utilizing models of the environment to improve the efficiency of learning, the thesis focuses on developing innovative control strategies using \ac{MBRL} for proficient gait regulation of soft quadruped robots, intending to promote progress in the field and introduce flexible, efficient robotic systems proficient in navigating complex environment autonomously.

\section{Problem Statement}
The research at hand adopts a novel approach by utilizing \ac{MFRL} for the development of innovative control strategies aimed at proficient movement regulation in soft quadruped robots\cite{jiSynthesizingOptimalGait2022}. Previous work has primarily focused on employing model-free reinforcement learning techniques to address challenges in larger state spaces, offering valuable insights into the potential of reinforcement learning for complex robotic systems. However, there remain several uncovered limitations that warrant meticulous consideration for future endeavors in this domain.

A prominent challenge is the significant time inefficiency inherent in training processes. The iterative nature of \ac{MFRL} and complecity of simulation to soft robot can lead to protracted training times, hindering the real-time application of learned strategies\cite{jiSynthesizingOptimalGait2022}. To address this challenge, innovative approaches have been explored. One such solution is the concept of \ac{MBRL}, which offers a promising approach to mitigate the time efficiency problem. \ac{MBRL} is considered to generate a functional representation of the robot's interaction with the external environment, so as to resemble the physical plant model and provide the state update feedback with high accuracy efficiently\cite{rayModelBasedReinforcementLearning2010}. In \ac{MBRL}, the agent learns a surrogate model of the environment, which can be used to make predictions about the future state of the environment and the possible outcomes of different actions, which allows the agent to plan ahead and make more informed decisions, allowing it to learn more efficiently and solve tasks more effectively\cite{polydorosSurveyModelBasedReinforcement2017}. The surrogate model within an MBRL framework to simulate the environment and predict the outcomes of different actions, rather than directly interacting with the real environment. This approach is considered to speed up learning and improve efficiency.

Furthermore, the complex dynamics of high-dimensional state and action spaces pose a formidable obstacle. The intricate interactions between the soft robot's flexible structure and its environment create a complex learning landscape\cite{arulkumaranDeepReinforcementLearning2017}, making exploration and convergence slower. In specific, the input to the \ac{RL} agent consists of state space and action space, and the state space was defined by available sensor measurements, including robot moving velocity in three directions, rotational angle in three directions and normalized contact force on four feet. The action space was defined by motors on the legs, and three motors on each leg. Therefore, the state-action space of the reinforcement learning reaches 22 dimensions, which expands the computational requirements and leads to sparsity in the reward function. As discussed in the background, the state space of the surrogate model consists of state transitions states and reward function states, which will also reach a significant high dimensions. To address this limitation, exploring dimensionality reduction techniques, such as advanced feature extraction methods\cite{polydorosSurveyModelBasedReinforcement2017} or learned representations\cite{wangBenchmarkingModelBasedReinforcement2019}, could offer more concise and effective representations of the state space, thereby enabling quicker and more adaptive learning. 

To address these challenges, this research employs approaches such as pattern-defined reinforcement learning and parameterization. In the previous training\cite{jiSynthesizingOptimalGait2022}, the gait controllers were learnt from actuator-level, but a popular way to design the locomotion controller in rigid quadruped robots is to define certain gait pattern, includes trot, pace, bound, pronk, gallop, etc.\cite{zhongAnalysisResearchQuadruped2019}. Therefore, if the gait controllers could be defined on certain patterns, the gait controllers could be simplified by some certain gaits and some actions could be composed to reduce the state-action space so as to increase the learning efficiency\cite{owakiQuadrupedRobotExhibiting2017}. For instance, the widely adopted trot, known for stability and balance\cite{fletcherTrot2012}, is chosen to restrict the movement of soft quadruped robots in reinforcement learning. Another method to restrict the state space of the plant model is parameterization, which parameterizing gait phases and control policies based on the abstraction of a quadruped robot\cite{shaoLearningFreeGait2022}, as demonstrated in previous work\cite{jiOmnidirectionalWalkingQuadruped2022}.

In conclusion, the investigation centers on enhancing the control strategies of soft quadruped robots, bridging the gap between learned strategies and real-time application. The research seeks to contribute to the advancement of robotics by addressing the challenges posed by time efficiency and dimensionality while harnessing the potential of reinforcement learning in the context of soft robotic locomotion control. The subsequent section outlines the research questions framed to guide this study's exploration and investigation.
\subsection*{Research Questions}
To elaborate, the problem addressed in this thesis was initially formulated by a set of research questions, which aimed to identify and explore the challenges and opportunities associated with gait control of soft quadruped robots. The following questions were designed to guide the research process and help frame the problem in a meaningful and relevant way.
\begin{enumerate}
    \item \label{rq1}How to restrict the state space or design a surrogate model with high estimation accuracy of soft quadruped robots plant compared to using Simulink Multibody functions? This model could be extracted as a representation of the real system, allowing for efficient and accurate simulations and training. Some methods considered to restrict the state space:
    \begin{enumerate}
        \item Pattern-defined reinforcement learning, it involves selecting a subset of features based on the certain pattern of quadruped robots, i.e. trot.
        \item Parameterization, the higher-level abstractions of the state-action space, it will use phases between gait and real motors to parameterize gaits of quadruped robots.
    \end{enumerate}
    \item \label{rq2}In comparison to model-free \ac{RL}, to what extend can the model-based \ac{RL} approach generate a better \ac{RL} agent and enhance the ability of a soft quadruped robot to walk in terms of stability, walking speed, and cost-of-transport? The enhancement than model-free \ac{RL} is a benchmark of this project. Furthermore, it is important to evaluate the performance of model-based \ac{RL}, and the evaluation of performance of this project focus on the simulation benefices, so what the trade-off is among the learning efficiency, the simulation accuracy and the long-term planning accuracy in order to train an optimal policy for gait control of soft quadruped robot?
\end{enumerate}
The ultimate goal of this thesis was to advance our understanding of gait control for soft quadruped robots and contribute to the development of effective strategies for controlling their motion. This was achieved by addressing a set of research questions, the answers to which are presented in Chapter \ref{chap6}. 

\section{Scope}
The objective of this study is to investigate and improve the learning efficiency associated with the current methods utilized for an optimal gait control of soft quadruped robots. This thesis proposes a \ac{MBRL} approach to improve the learning efficiency and accuracy of optimal gait control while developing resilient and efficient gait control policies that can handle the continuous and deformable morphology of the robot. Moreover, the research seeks endeavors to make a valuable contribution to the development of more efficient and effective \ac{RL} control strategies that can facilitate soft quadruped robots to perform tasks in real-world scenarios. To validate these concepts, practical physical tests are conducted to corroborate the research's practical applicability and effectiveness. Ultimately, this thesis also introduces a new software architecture tailored to the specific requirements of the  soft quadruped robot SoftQ.

\subsection*{Limitations}
The research is encumbered by the concept of the "sim-to-real gap," representing the disparities between simulated and real-world scenarios, Simulations and experiments conducted within controlled environments possess limitations in accurately emulating the complexities of real-world contexts. As a consequence, the transition of simulated models to practical applications can result in performance discrepancies. This inherent limitation arises due to the potential divergence between simulation outcomes and real-world behaviors, potentially rendering conclusions derived from simulations inapplicable in real-world contexts. Additionally, the interconnected nature of the robot's legs imposes constraints on the diversity of gaits that can be studied, restricting the robot's capability for certain types of locomotion requiring individual leg movement. This limitation inhibits the exploration of locomotion strategies relevant to real-world scenarios. Consequently, the evaluation of controllers will not focus on algorithm intricacies but rather on the impact of these controllers on the robot's stability, walking speed, and cost-of-transport.

\subsection*{Delimitation}
This thesis is centered around a thorough exploration and resolution of the challenges associated with model-based reinforcement learning in the context of optimal gait control for soft quadruped robots. However, it is essential to acknowledge and define specific delimitation that provide a scope to the study. These delimitation shape the boundaries within which the research operates and offers a clear understanding of the study's focus. Firstly, this research is delimited to the examination of soft quadruped robots exclusively and does not encompass other categories of robots or diverse robotic systems. The study narrows its scope intentionally to maintain a concentrated and in-depth analysis of the challenges pertinent to soft quadruped robots. Secondly, the influence of external factors, such as wind, terrain variations, and obstacles, is deliberately excluded from consideration within this study. While these external conditions can significantly impact the performance of robotic systems, their exclusion is necessary to maintain a focused exploration of the model-based reinforcement learning challenges specific to gait control. Furthermore, it is important to highlight that this project operates under the assumption that the hardware design of the soft quadruped robot is both fixed and operational. Variations or changes in the hardware design are not considered within the scope of this research. Finally, the project's scope is delimited to a specific model-based algorithm for reinforcement learning, i.e. \ac{SAC}. Alternative approaches, methodologies, or algorithms for gait control are intentionally not explored within this study. The research is designed to deeply investigate the intricacies and potential solutions within the context of the chosen model-based algorithm.

\section{Methodology}
In this degree project, a set of methodologies and methods have been employed to address the research questions and achieve the objectives. The methodology employed comprises four main stages, namely data collection and processing, model-based \ac{RL} algorithm development and comparison, evaluation and analysis, and validation. Detailed description of these methodologies and methods will be presented in Chapter \ref{chap3}. 

Firstly, the signals to motors from the typical trot of quadruped robots were extracted by studying the actuation of the soft quadruped robot trot pattern. Subsequently, a model of the soft quadruped robot was obtained and used to generate simulation data using the Simscape model based on the extracted state space. A surrogate model was then designed with high estimation accuracy of the soft quadruped robot plant based on the processed data, which could effectively simulate the dynamics of the system and train an optimal policy for gait control. After that, a model-based reinforcement learning algorithm was developed for gait control of the soft quadruped robot using the extracted features and the reduced state space. The algorithm's performance was evaluated in simulation using metrics such as stability, walking speed, and cost-of-transport, and its design and parameters were iteratively modified to improve its performance. The previous model-free reinforcement learning algorithm's performance was also evaluated and compared using the same metrics. In the next step, the trade-offs between learning efficiency, simulation accuracy, and long-term planning accuracy were evaluated in the context of training an optimal policy for gait control of the soft quadruped robot. The results were analyzed to draw conclusions about the effectiveness of the proposed model-based reinforcement learning approach compared to model-free reinforcement learning. Finally, the proposed approach was validated by implementing the optimal policy on the physical soft quadruped robot and measuring its performance in a real-world setting in terms of walking speed, stability, and cost-of-transport. The physical robot's performance was compared with the simulated results to validate the accuracy of the simulation and the effectiveness of the proposed approach. The graphical representation depicted in the Figure \ref{fig:method} provides a comprehensive outline of this thesis and enumerates the areas of investigation that have been explored.
\tikzstyle{chevron}=[shape=signal, draw, signal from=west, signal to=east,
    align=center, font=\small, minimum height=3em, draw, minimum width=4em, 
    node distance = 0.5em, inner xsep=1em]

\begin{figure}[ht]
    \centering
    \begin{tikzpicture}[auto]
    \node[chevron, start chain=going right](1){Model design};
    \node[chevron, right=0.5em of 1, on chain](2){\ac{MBRL} algorithm \\development};
    \node(3)[chevron, on chain]{Experiments and \\analysis};
    \node(4)[chevron, on chain]{Validation};
    \node[font=\footnotesize, below=0em of 1,text width=4cm]{-Feature extraction\\ -Data collection\\ -Surrogate model training};
    \node[font=\footnotesize, below=0em of 2,text width=4cm]{-Model implementation\\ -\ac{MBRL} algorithm design\\ -Validate \ac{MBRL} algorithm};
    \node[font=\footnotesize, below=0em of 3,text width=4cm]{-\ac{MFRL} algorithm test\\ -Experiments\\ -Analysis};
    \node[font=\footnotesize, below=0em of 4,text width=4cm]{-Validation on real robot\\ -Evaluation of performance};
    \end{tikzpicture}
    \caption{Overview of the methods in this thesis}
    \label{fig:method}
\end{figure}
 
\section{Ethics and Sustainability}
The use of robotics and artificial intelligence raises ethical and sustainability considerations that need to be addressed in this project. One ethical consideration is related to the potential for the quadruped robot to be used for military or surveillance purposes, which could have negative impacts on privacy and human rights. To ensure that the robot is used ethically, the project will focus on the development of an optimal gait controller for soft quadruped robots for use in research and other non-military applications. Sustainability considerations include the environmental impact of the materials used in the construction of the robot, as well as the potential impact of the project on the environment through energy consumption and waste. To address these considerations, the project will use environmentally-friendly materials where possible and focus on energy efficiency in the design and testing of the robot.

\section{Outline}
This paper is structured into 6 chapters, beginning with an introductory Chapter \ref{chap1} that provides readers with necessary contextual information and a clear overview of this thesis. Chapter \ref{chap2} is dedicated to a comprehensive review of existing literature in the field, with a particular focus on defining key concepts and providing a thorough problem description. This chapter also deals with the design of soft continuum actuator and the design of the basic structure used to train the neural networks and \ac{RL} agent. Chapter \ref{chap3} outlines the research methods utilized to answer the research questions. Chapter \ref{chap4} is devoted to the implementation of detailed design and the experiments conducted for the research questions. Chapter \ref{chap5} presents the results of the experiments, which also includes the major modifications made to the existing \ac{RL} training method for the improvement. Finally, Chapter \ref{chap6} provides a comprehensive summary of the research project and its future prospects.
