\chapter{Methods and Methodologies}
\label{chap3}
% \thispagestyle{fancy}
\textit{This chapter motivates the methods and methodologies to answer the research questions.}
Describe the engineering-related contents (preferably with models) and the research methodology and methods that are used in the degree project. 

Most likely it generally describes the method used in each step to make sure that you can answer the research question.

\section{Methodologies}
Concerning \hyperref[rq1]{research question 1}, the state space restriction methods discussed should be confident to restrict the state space and increase the learning efficiency, since the algorithm is constructed directly from the state space, but the effectiveness of the surrogate model is undetermined. Therefore, an assessment of the effectiveness of the surrogate model is required, taking into account key metrics such as the Root Mean Squared Error (RMSE) of prediction on the gaits, Coefficient of determination ($R^2$) between simulations and predictions on the reference, and Normalized Root Mean Squared Error (NRMSE) of observations from the sensors on the robot. Then, the independent variables that will be used are settings of parameterization parameters, rotational angle $\alpha_r$, bending angle $\alpha_b$ and compressed length of the actuator $z_l$, which was determined by previous parameterization on the continuum actuators\cite{jiOmnidirectionalWalkingQuadruped2022}. Thus, dependent variables are simulation time, prediction accuracy and long-term prediction accuracy, where long-term prediction 
\begin{itemize}
    \item Independent variables: 
    \begin{itemize}
        \item Compressed length of the actuator limit($z_l$), Type: Continuous, Units: Millimeters
        \item Rotational angle ($\alpha_r$), Type: Continuous, Units: Radian
        \item Bending angle limit ($\alpha_b$), Type: Continuous, Units: Radian
    \end{itemize}
    \item Dependent variables:
    \begin{itemize}
        \item Model accuracy, Type: Continuous, Units: RMSE, $R^2$, RRMSE, Percentage
        \item Long-term prediction accuracy, Type: Continuous, Units: Percentage
        \item Simulation time, Type: Continuous, Units: Second
    \end{itemize}
\end{itemize}

After the data was collected, the data should be preprocess to conduct a correlation analysis to determine if there is a linear relationship between the independent variables and the dependent variable. To analyze the data, three multiple linear regressions will be conducted, with the three parameters as independent variables and each of performance metrics as the dependent variable. In addition, a best performane model should be determined from this test and proceed it to answer research question 2.
Regarding to \hyperref[rq2]{research question 2}, another ANOVA test could be conducted to evaluate the performance of model-based RL agent in comparison to model-free RL agent in terms of stability, walking speed and cost-of-transport. Firstly, the independent variables are defined as the desired walking speed and the type of RL method, with two levels: model-based RL and model-free RL. The dependent variables will be stability, walking speed, and cost of transport, measured as performance metrics during the gait. The stability of robot could be quantified by the zero-moment point (ZMP) method, while walking speed and cost-of-transport are direct performance metrics during gait. Data will be collected by simulating the gait controller using both RL methods, and measuring the three performance metrics for each simulation run, resulting in three data sets for each RL method. 
\begin{itemize}
    \item Independent variables: 
    \begin{itemize}
        \item Type of RL method, Type: Categorical, Units: Model-based, Model-free
    \end{itemize}
    \item Covariance:
    \begin{itemize}
        \item Desired walking speed ($v_x$), Type: Continuous, Units: Meter per second
    \end{itemize}
    \item Dependent variables:
    \begin{itemize}
        \item Stability, Type: Continuous, Units: Sum of moment, Newton*meter
        \item Resultant walking speed, Type: Continuous, Units: Meter per second
        \item Cost-of-transport, Type: Continuous, Units: Percentage
        \item Learning efficiency, Type: Continuous, Units: Number of iterations to converge
        \item Long-term planning , Type: Continuous, Units: Cumulative reward
    \end{itemize}
\end{itemize}
A one-way ANCOVA test will be conducted with RL method type as one factor, and the desired walking speed as a covariance. The null hypothesis is that there is significant difference in the means of the performance metrics between the two RL methods with different desired walking speed. If the null hypothesis is rejected, it indicates that there is a significant difference in at least one of the performance metrics between the two RL methods. The test bed is listed in Table \ref{tab:rq2test}. Furthermore, a Pareto analysis will be employed to evaluate the trade-off between multiple objectives, focusing on simulation benefits, where learning efficiency can be measured by the number of iterations or episodes required for the model-based RL algorithm to converge to a satisfactory policy, long-term planning effectiveness could be measured by evaluating cumulative reward obtained by the policy over a longer horizon. The Pareto front is then determined by identifying the set of solutions that cannot be improved in one objective without worsening at least one other objective. The Pareto set can be also computed, which is the set of algorithms that correspond to the Pareto front, and the Pareto optimal solutions will also be determined, which are the objective values associated with each point on the Pareto front. 
\begin{longtable}{|p{1cm}|cccccccc|}
\hline
    \multicolumn{1}{|c|}{\multirow{2}{*}{Method}} &
      \multicolumn{8}{c|}{\begin{tabular}[c]{@{}c@{}}Desired walking\\  speed $v_x$ (m/s)\end{tabular}} \\ \cline{2-9} 
    \multicolumn{1}{|c|}{} &
      \multicolumn{1}{c|}{0.01} &
      \multicolumn{1}{c|}{0.1} &
      \multicolumn{1}{c|}{0.3} &
      \multicolumn{1}{c|}{0.5} &
      \multicolumn{1}{c|}{0.75} &
      \multicolumn{1}{c|}{1} &
      \multicolumn{1}{c|}{1.5} &
      3 \\ \hline
    \endfirsthead
    %
    \endhead
    %
    Model-based RL &
      \multicolumn{1}{c|}{\begin{tabular}[c]{@{}c@{}}Test 1:\\ Test 2:\\ Test 3:\end{tabular}} &
      \multicolumn{1}{c|}{\begin{tabular}[c]{@{}c@{}}Test 1:\\ Test 2:\\ Test 3:\end{tabular}} &
      \multicolumn{1}{c|}{\begin{tabular}[c]{@{}c@{}}Test 1:\\ Test 2:\\ Test 3:\end{tabular}} &
      \multicolumn{1}{c|}{\begin{tabular}[c]{@{}c@{}}Test 1:\\ Test 2:\\ Test 3:\end{tabular}} &
      \multicolumn{1}{c|}{\begin{tabular}[c]{@{}c@{}}Test 1:\\ Test 2:\\ Test 3:\end{tabular}} &
      \multicolumn{1}{c|}{\begin{tabular}[c]{@{}c@{}}Test 1:\\ Test 2:\\ Test 3:\end{tabular}} &
      \multicolumn{1}{c|}{\begin{tabular}[c]{@{}c@{}}Test 1:\\ Test 2:\\ Test 3:\end{tabular}} &
      \begin{tabular}[c]{@{}c@{}}Test 1:\\ Test 2:\\ Test 3:\end{tabular} \\ \hline
    Model-free RL &
      \multicolumn{1}{c|}{\begin{tabular}[c]{@{}c@{}}Test 1:\\ Test 2:\\ Test 3:\end{tabular}} &
      \multicolumn{1}{c|}{\begin{tabular}[c]{@{}c@{}}Test 1:\\ Test 2:\\ Test 3:\end{tabular}} &
      \multicolumn{1}{c|}{\begin{tabular}[c]{@{}c@{}}Test 1:\\ Test 2:\\ Test 3:\end{tabular}} &
      \multicolumn{1}{c|}{\begin{tabular}[c]{@{}c@{}}Test 1:\\ Test 2:\\ Test 3:\end{tabular}} &
      \multicolumn{1}{c|}{\begin{tabular}[c]{@{}c@{}}Test 1:\\ Test 2:\\ Test 3:\end{tabular}} &
      \multicolumn{1}{c|}{\begin{tabular}[c]{@{}c@{}}Test 1:\\ Test 2:\\ Test 3:\end{tabular}} &
      \multicolumn{1}{c|}{\begin{tabular}[c]{@{}c@{}}Test 1:\\ Test 2:\\ Test 3:\end{tabular}} &
      \begin{tabular}[c]{@{}c@{}}Test 1:\\ Test 2:\\ Test 3:\end{tabular} \\ \hline
    
    
      \caption{ANCOVA tests examples for research question 2}
      \label{tab:rq2test}
\end{longtable}

Applying engineering related and scientific skills; modelling, analysing, developing, and evaluating engineering-related and scientific content; correct choice of methods based on problem formulation; consciousness of aspects relating to society and ethics (if applicable).

\section{Methods}
In this degree project, a set of methodologies and methods have been employed to address the research questions and achieve the objectives. The methodology employed comprises four main stages, namely data collection and processing, model-based \ac{RL} algorithm development and comparison, evaluation and analysis, and validation. Detailed description of these methodologies and methods will be presented in Chapter \ref{chap3}. Firstly, the signals to motors from the typical trot of quadruped robots were extracted by studying the actuation of the soft quadruped robot trot pattern. Subsequently, a model of the soft quadruped robot was obtained and used to generate simulation data using the Simscape model based on the extracted state space. A surrogate model was then designed with high estimation accuracy of the soft quadruped robot plant based on the processed data, which could effectively simulate the dynamics of the system and train an optimal policy for gait control. Secondly, a model-based reinforcement learning algorithm was developed for gait control of the soft quadruped robot using the extracted features and the reduced state space. The algorithm's performance was evaluated in simulation using metrics such as stability, walking speed, and cost-of-transport, and its design and parameters were iteratively modified to improve its performance. The previous model-free reinforcement learning algorithm's performance was also evaluated and compared using the same metrics. Thirdly, the trade-offs between learning efficiency, simulation accuracy, and long-term planning accuracy were evaluated in the context of training an optimal policy for gait control of the soft quadruped robot. The results were analyzed to draw conclusions about the effectiveness of the proposed model-based reinforcement learning approach compared to model-free reinforcement learning. Finally, the proposed approach was validated by implementing the optimal policy on the physical soft quadruped robot and measuring its performance in a real-world setting in terms of walking speed, stability, and cost-of-transport. The physical robot's performance was compared with the simulated results to validate the accuracy of the simulation and the effectiveness of the proposed approach. The graphical representation depicted in the Figure \ref{fig:method} provides a comprehensive outline of this thesis and enumerates the areas of investigation that have been explored.
%
\tikzstyle{chevron}=[shape=signal, draw, signal from=west, signal to=east,
    align=center, font=\small, minimum height=3em, draw, minimum width=4em, 
    node distance = 0.5em, inner xsep=1em]

\begin{figure}[hp]
    \centering
    \begin{tikzpicture}[auto]
    \node[chevron, start chain=going right](1){Model design};
    \node[chevron, right=0.5em of 1, on chain](2){\ac{MBRL} algorithm \\development};
    \node(3)[chevron, on chain]{Evaluation and \\analysis};
    \node(4)[chevron, on chain]{Validation};
    \node[font=\footnotesize, below=0em of 1,text width=4cm]{-Feature extraction\\ -Data collection\\ -Surrogate model training};
    \node[font=\footnotesize, below=0em of 2,text width=4cm]{-Model implementation\\ -\ac{MBRL} algorithm design\\ -Validate \ac{MBRL} algorithm};
    \node[font=\footnotesize, below=0em of 3,text width=4cm]{-\ac{MFRL} algorithm test\\ -Experiments\\ -Analysis};
    \node[font=\footnotesize, below=0em of 4,text width=4cm]{-Validation on real robot\\ -Evaluation of performance};
    \end{tikzpicture}
    \caption{Overview of the methods in this thesis}
    \label{fig:method}
\end{figure}

As mentioned earlier, give a theoretical description of methodologies and methods and how these are applied in the degree project.
