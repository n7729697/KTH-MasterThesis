\chapter{Discussions and Conclusions}
\label{chap6}
\textit{This chapter summarizes the key findings and contributions of this thesis, providing insights into the achievements and implications of the research conducted. It also outlines the limitations of this thesis and suggests potential avenues for future research.}

\section{Discussion}
In summary, this research endeavor has sought to unravel the complexities surrounding gait control in soft quadruped robots, with the goal of enhancing learning efficiency by Model Based Reinforcement Learning (MBRL) for more effective control strategies. A systematic exploration of the research questions outlined in this study has yielded valuable insights and findings that contribute significantly to the field of robotics and autonomous locomotion. Additionally, this section will conduct a comprehensive discussion of the research questions outlined in the introduction, with the aim of providing answers and insights based on the research conducted in this thesis.

\subsection{Answer to Research Questions 1}
The investigation into restricting the state space and designing a surrogate model with high estimation accuracy was instrumental in the pursuit of efficient and accurate simulations and training for soft quadruped robots. This research sought to address this question through two distinct approaches: pattern-defined reinforcement learning and parameterization. These methods provide a solid foundation for efficient and accurate simulations and training, ultimately leading to improved gait control strategies, as shown in Section \ref{Sec:MBRL}. 

The concept of pattern-defined reinforcement learning showed promise as a means to restrict the state space efficiently. By selecting a subset of features based on specific patterns, such as the trot, the study aimed to simplify the state space while retaining essential information. This approach represents a significant step towards efficient simulations and training.

Parameterization introduced a higher-level abstraction into the state-action space by using phases between gait and real motors to parameterize the gaits of quadruped robots. This approach, while complex, holds the potential to simplify the representation of the state space. It has the advantage of capturing critical information about locomotion while streamlining the learning process.

Additionally, it's important to acknowledge the inherent challenges posed by the interconnected nature of the robot's legs. This unique characteristic of soft quadruped robots introduces constraints on the diversity of gaits that can be studied. Unlike robots with independently movable legs, soft quadrupeds exhibit a more synchronized form of locomotion due to their interconnected structure. This limitation restricts the robot's capability to perform certain types of locomotion that may be relevant in real-world scenarios, where adaptability is key.

Given these constraints, our research approach pivoted towards a practical evaluation methodology that prioritized the tangible impact of controllers on the robot's performance in real-world applications. Rather than delving into the intricacies of algorithms in isolation, we directed our focus towards assessing how these controllers influence the robot's stability, walking speed, and cost-of-transport. These metrics serve as practical indicators of a controller's effectiveness in real-world contexts, aligning our research with the practical needs of the field.

This shift in emphasis allowed us to bridge the gap between theoretical developments and real-world deployment, ensuring that our research contributes directly to the enhancement of soft quadruped robots' capabilities in practical scenarios. In essence, our approach recognized the correlation of algorithm design and real-world performance, striving for a holistic understanding of gait control in these unique robotic systems.

\subsection{Answer to Research Questions 2}
Furthermore, the comparative analysis between model-based and model-free reinforcement learning approaches has shed light on the potential of the former in generating superior control agents for soft quadruped robots.In conclusion, the comparative analysis conducted in response to Research Question 2 has highlighted the potential superiority of model-based reinforcement learning over its model-free counterpart in the realm of soft quadruped robot control. The observed improvements in terms of stability, walking speed, and cost-of-transport underscore the significant role played by model-based reinforcement learning as a benchmark within the scope of this project.

Regarding stability, our findings indicate that both the reinforcement learning method and the parameter $v_{ref}$ have a statistically significant impact on the robot's walking stability. Moreover, the interaction between these factors also proved to be significant, suggesting that the choice of the most effective reinforcement learning method may vary depending on the specific value of $v_{ref}$. Notably, the reinforcement learning method exhibited a statistically significant influence on learning efficiency, while $v_{ref}$ did not. Furthermore, the interaction between these two factors was found to be borderline significant, emphasizing the consistent effect of the reinforcement learning method across different $v_{ref}$ levels.

Additionally, our research has brought to light the importance of carefully considering the trade-offs among learning efficiency, simulation accuracy, and long-term planning accuracy when striving to train an optimal gait control policy for soft quadruped robots. This comprehensive examination of trade-offs has provided invaluable insights that contribute to the quest for optimal gait control policies.

One noteworthy aspect of our approach lies in its adaptability, allowing for the customization of the controller's behavior to suit various task objectives by manipulating distinct reward functions within the reinforcement learning framework. As the reinforcement learning algorithm learns to optimize the controller for different tasks based on specific reward functions, this adaptability renders our methodology versatile and applicable to a wide range of scenarios. Consequently, the robotic system can acquire a diverse set of task-specific skills, enhancing its overall versatility and capability.

\section{Conclusion}
In conclusion, this thesis embarked on an exploration of enhancing learning efficiency in the context of optimal gait control for soft quadruped robots, with a primary emphasis on addressing the challenges through a MBRL approach. The overall objective was to develop robust gait control policies capable of accommodating the continuous and deformable morphology of the robot while contributing to the advancement of RL control strategies for real-world applications. To substantiate the theoretical framework and proposed methodologies, a series of experimental tests were conducted, ensuring the practical applicability of the research findings. Additionally, a tailored software architecture designed specifically for the soft quadruped robot, SoftQ, was introduced to facilitate the research's objectives.

Nonetheless, it is imperative to acknowledge certain limitations and delimitations that influenced the thesis's scope. The "sim-to-real gap" constraint underscored the challenges associated with translating simulation results to real-world scenarios, reinforcing the need for a practical evaluation of the proposed gait control strategies. Moreover, the interconnected nature of the robot's legs imposed constraints on the variety of gaits that could be studied, limiting the exploration of certain locomotion strategies relevant to real-world contexts. These limitations necessitated an evaluation focused on the practical impact of controllers on the robot's stability, walking speed, and cost-of-transport, rather than delving into algorithm intricacies.

Delimitations were intentionally set to maintain a concentrated focus on soft quadruped robots, excluding other robotic categories, external environmental factors, hardware design variations, and alternative RL algorithms. This deliberate narrowing of scope aimed to foster an in-depth analysis of the specific challenges related to gait control in the chosen context.

The research methodology encompassed stages of data collection, model-based RL algorithm development, evaluation, analysis, and validation, culminating in the comparison of model-based and model-free RL approaches. Practical validation through physical implementation on SoftQ ensured the relevance of the findings in real-world scenarios. The project also acknowledged ethical and sustainability considerations, emphasizing the responsible use of robotics technology and environmentally-conscious design choices.

In essence, this thesis contributes to the advancement of soft quadruped robotics by proposing an MBRL approach to gait control, with practical validation underscoring its potential for real-world applications. It is anticipated that the insights gained from this research will pave the way for more efficient and effective RL control strategies, ultimately enhancing the capabilities of soft quadruped robots in various domains, while remaining mindful of ethical and sustainability concerns.

\section{Future Work}
This thesis has explored the context of optimal gait control for soft quadruped robots, uncovering valuable insights and paving the way for further investigations. As with any scientific endeavor, there are promising avenues for future research that can expand upon the knowledge and limitations identified in this study. In this section, the potential directions of future research were outlined for the advancement of soft quadruped robot system.

\subsubsection*{Hardware advancements}
The field of soft robotics is continually evolving, with ongoing developments in soft actuators, materials, and sensors. Future research can explore the integration of state-of-the-art hardware components to enhance the physical capabilities and adaptability of quadruped robots. The continuous evolution of soft robotic materials and actuators presents an opportunity for future research. Investigating the integration of cutting-edge soft robotics hardware, such as novel actuators and sensors\cite{tanShapeEstimation3D2022,tanEdgeEnabledAdaptiveShape2024}, can enhance the physical capabilities and adaptability of quadruped robots, thereby advancing their locomotion control.

\subsubsection*{Multi-terrain adaptation}
Soft quadruped robots should be able to navigate a wide range of terrains, from rugged outdoor environments to indoor spaces. Future research should prioritize the development of control algorithms that enable seamless adaptation to various surfaces. This includes rough, uneven terrains, dynamically changing landscapes, and transitions between different terrain types. Achieving robust multi-terrain adaptation will significantly expand the practical utility of soft quadruped robots across diverse applications.

\subsubsection*{Online learning and adaptation}
The ability of robots to adapt to changing environmental conditions and unforeseen challenges in real-time is crucial. Future research can focus also on developing online learning and adaptation mechanisms that allow soft quadruped robots to continuously update their control policies during operation. This real-time adaptability will be essential for improving performance, responsiveness, and autonomy in dynamic environments.

\subsubsection*{Feedback control to achieve position control}
Incorporating feedback control mechanisms to achieve precise position control is an important area of exploration. Future research can delve into advanced control strategies that enable soft quadruped robots to achieve and maintain specific positions. The significance of this lies in tasks that demand meticulous positioning, such as precise robotic manipulation, obstacle avoidance, and navigating through confined spaces. By developing control techniques that seamlessly incorporate precise position control into the skill set of soft quadruped robots, we can open up new possibilities for their application in a wider array of scenarios.

\subsubsection*{Robustness and fault tolerance}
Ensuring the robustness and fault tolerance of soft quadruped robots is critical for real-world applications. Future research can investigate robust control strategies capable of withstanding disturbances, uncertainties, and unexpected environmental changes. Additionally, the development of fault-tolerant control mechanisms will be essential for handling hardware failures or damage, ensuring the reliability and safety of these robots in challenging scenarios.

\subsection{Sustainability and Ethical Considerations}
In a world increasingly concerned with environmental impact, the notion of environmental sustainability is more relevant than ever. In the context of this thesis, significant consideration was given to the COT metric as a guiding factor in the design of the controller. Emphasis was placed on the role of energy efficiency in the performance of soft quadruped robots, with the objective of creating controllers that enhance the robot's endurance while minimizing its environmental impact.

The choice of materials and manufacturing techniques applied in this research proved to be highly advantageous in achieving these goals. The utilization of compliant and soft materials was aligned with the inherent characteristics of quadruped locomotion and contributed to improved energy efficiency. These materials facilitated better energy absorption and recovery during the robot's gait, leading to reduced energy losses and enhanced overall efficiency.

Furthermore, the incorporation of advanced 3D printing methods enabled the creation of intricate and customized components for the soft quadruped robot. This level of manufacturing precision not only minimized material wastage but also bolstered structural integrity, further supporting the overarching objectives of sustainability and energy efficiency. These considerations are not mere addenda but foundational principles that should guide every stage of robotic development.

Crucially, regulatory frameworks serve as the scaffolding upon which sustainability and ethics are built. Collaborative efforts among governments, industry bodies, and researchers are essential to craft standards and guidelines that embody these principles. Such frameworks are not restrictive but liberating, providing a clear roadmap for the responsible development and deployment of soft quadruped robots.

As soft quadruped robots become integrated into society, ethical considerations, including privacy, safety, and their impact on employment, require careful examination. Future research should also delve into these ethical dimensions to ensure the responsible deployment of these robots.

\subsection{Final Words}
In these final words, it is crucial to acknowledge the collaborative efforts of researchers, engineers and innovators who continue to push the boundaries of robotics. The journey to master gait control in soft quadruped robots is ongoing, with numerous challenges and opportunities awaiting exploration.

In the grand scheme of scientific exploration, this research represents a small but significant stride toward unlocking the full potential of soft quadruped robots. It is our hope that the knowledge gained here will inspire further research and innovation, ultimately leading to the realization of highly capable and versatile soft quadruped robots that can navigate the complexities of our ever-changing world.

In closing, this thesis marks not an end, but rather a new beginning in the journey to harness the untapped potential of soft quadruped robots. May our collective efforts continue to drive progress, pushing the boundaries of what these remarkable machines can achieve.






