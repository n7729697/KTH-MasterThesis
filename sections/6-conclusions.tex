\chapter{Conclusions}
\label{chap6}
\textit{This chapter summarizes the key findings and contributions of this thesis, providing insights into the achievements and implications of the research conducted. It also outlines the limitations of the study and suggests potential avenues for future research.}

\section{Discussion}
In conclusion, this research endeavor has sought to unravel the complexities surrounding gait control in soft quadruped robots, aiming to enhance our understanding and pave the way for more effective control strategies. Through a systematic exploration of the research questions outlined in this study, we have unearthed valuable insights and findings that contribute to the field of robotics and autonomous locomotion.

Our investigation into methods for restricting the state space and designing surrogate models with high estimation accuracy has revealed innovative approaches such as pattern-defined reinforcement learning and parameterization. These methods provide a solid foundation for efficient and accurate simulations and training, ultimately leading to improved gait control strategies.

Furthermore, the comparative analysis between model-based and model-free reinforcement learning approaches has shed light on the potential of the former in generating superior control agents for soft quadruped robots. The enhancements in stability, walking speed, and cost-of-transport underscore the significance of model-based reinforcement learning as a benchmark in this project. The careful consideration of trade-offs among learning efficiency, simulation accuracy, and long-term planning accuracy has provided essential insights into the quest for optimal gait control policies.

\section{Future Work}

An intriguing fact of our approach resides in the capacity to tailor the controller's behavior to various task objectives through the manipulation of distinct reward functions within the RL framework. By delineating specific reward functions, the RL algorithm learns to optimize the controller for diverse tasks of interest. This adaptability renders the methodology versatile and applicable to a spectrum of scenarios, allowing the robotic system to be endowed with a repertoire of task-specific skills.

This thesis has ventured into the realm of gait control for soft quadruped robots, uncovering valuable insights and paving the way for further investigations. As with any scientific endeavor, there are promising avenues for future research that can expand upon the knowledge and limitations identified in this study. In this section, we outline potential directions for future research in the field of soft quadruped robot gait control.

1. Real-world Validation

While this research primarily focused on simulation-based gait control, an imperative direction for future research involves the validation of control strategies in real-world scenarios. Conducting experiments with physical soft quadruped robots can bridge the gap between simulated environments and practical applications, enabling a more comprehensive assessment of control strategies and their effectiveness in real-world conditions.

2. Hardware Advancements

The continuous evolution of soft robotic materials and actuators presents an opportunity for future research. Investigating the integration of cutting-edge soft robotics hardware, such as novel actuators and sensors, can enhance the physical capabilities and adaptability of quadruped robots, thereby advancing their locomotion control.

3. Hybrid Control Strategies

Future research can explore the development of hybrid control strategies that combine the strengths of model-based and model-free reinforcement learning techniques. Such an approach may leverage the accuracy of model-based methods while retaining the adaptability and robustness of model-free techniques, offering a more versatile and reliable solution for gait control.

4. Multi-Terrain Adaptation

Soft quadruped robots are designed to navigate diverse terrains. Future research should prioritize the development of control algorithms that enable seamless adaptation to various surfaces, including rough, uneven, and dynamically changing terrains. This will expand the practical utility of soft quadruped robots across a wide range of environments.

5. Online Learning and Adaptation

Implementing online learning and adaptation mechanisms is crucial for enhancing a robot's ability to respond to changing environmental conditions and unforeseen challenges. Future research can focus on developing algorithms that allow soft quadruped robots to continuously update their control policies in real-time, improving their adaptability and performance.

6. Human-Robot Interaction

As soft quadruped robots find applications in human-centric environments, investigating human-robot interaction aspects becomes essential. Future research can explore how these robots can effectively collaborate with humans in tasks such as search and rescue, healthcare, logistics, and more, ensuring safe and seamless integration.

7. Energy Efficiency

Optimizing the energy efficiency of gait control is paramount, particularly for applications with limited battery resources. Future work should concentrate on developing control strategies that minimize energy consumption while maintaining or even enhancing performance, addressing a critical aspect of practical robot deployment.

8. Robustness and Fault Tolerance

Future research should delve into robust control strategies capable of withstanding disturbances and uncertainties in real-world scenarios. Additionally, investigating fault-tolerant control mechanisms to handle hardware failures or damage will be essential for ensuring the reliability of soft quadruped robots.

9. Multi-Robot Coordination

In scenarios involving multiple soft quadruped robots, coordinating their actions effectively is a complex challenge. Future research can explore techniques for multi-robot coordination, facilitating collaborative tasks in domains such as surveillance, exploration, and disaster response.

10. Ethical Considerations

As soft quadruped robots become integrated into society, ethical considerations, including privacy, safety, and their impact on employment, require careful examination. Future research should delve into these ethical dimensions to ensure the responsible deployment of these robots.

\subsection{Final Words}
In these final words, it is crucial to acknowledge the collaborative efforts of researchers, engineers and innovators who continue to push the boundaries of robotics. The journey to master gait control in soft quadruped robots is ongoing, with numerous challenges and opportunities awaiting exploration.

In the grand scheme of scientific exploration, this research represents a small but significant stride toward unlocking the full potential of soft quadruped robots. It is our hope that the knowledge gained here will inspire further research and innovation, ultimately leading to the realization of highly capable and versatile soft quadruped robots that can navigate the complexities of our ever-changing world.

In closing, this thesis marks not an end, but rather a new beginning in the journey to harness the untapped potential of soft quadruped robots. May our collective efforts continue to drive progress, pushing the boundaries of what these remarkable machines can achieve.






